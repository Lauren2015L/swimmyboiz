\documentclass[]{article}
\usepackage{lmodern}
\usepackage{amssymb,amsmath}
\usepackage{ifxetex,ifluatex}
\usepackage{fixltx2e} % provides \textsubscript
\ifnum 0\ifxetex 1\fi\ifluatex 1\fi=0 % if pdftex
  \usepackage[T1]{fontenc}
  \usepackage[utf8]{inputenc}
\else % if luatex or xelatex
  \ifxetex
    \usepackage{mathspec}
  \else
    \usepackage{fontspec}
  \fi
  \defaultfontfeatures{Ligatures=TeX,Scale=MatchLowercase}
\fi
% use upquote if available, for straight quotes in verbatim environments
\IfFileExists{upquote.sty}{\usepackage{upquote}}{}
% use microtype if available
\IfFileExists{microtype.sty}{%
\usepackage{microtype}
\UseMicrotypeSet[protrusion]{basicmath} % disable protrusion for tt fonts
}{}
\usepackage[margin=1in]{geometry}
\usepackage{hyperref}
\hypersetup{unicode=true,
            pdftitle={EEB313 Final Project - swimmyboiz},
            pdfauthor={Dianya Luo},
            pdfborder={0 0 0},
            breaklinks=true}
\urlstyle{same}  % don't use monospace font for urls
\usepackage{color}
\usepackage{fancyvrb}
\newcommand{\VerbBar}{|}
\newcommand{\VERB}{\Verb[commandchars=\\\{\}]}
\DefineVerbatimEnvironment{Highlighting}{Verbatim}{commandchars=\\\{\}}
% Add ',fontsize=\small' for more characters per line
\usepackage{framed}
\definecolor{shadecolor}{RGB}{248,248,248}
\newenvironment{Shaded}{\begin{snugshade}}{\end{snugshade}}
\newcommand{\KeywordTok}[1]{\textcolor[rgb]{0.13,0.29,0.53}{\textbf{#1}}}
\newcommand{\DataTypeTok}[1]{\textcolor[rgb]{0.13,0.29,0.53}{#1}}
\newcommand{\DecValTok}[1]{\textcolor[rgb]{0.00,0.00,0.81}{#1}}
\newcommand{\BaseNTok}[1]{\textcolor[rgb]{0.00,0.00,0.81}{#1}}
\newcommand{\FloatTok}[1]{\textcolor[rgb]{0.00,0.00,0.81}{#1}}
\newcommand{\ConstantTok}[1]{\textcolor[rgb]{0.00,0.00,0.00}{#1}}
\newcommand{\CharTok}[1]{\textcolor[rgb]{0.31,0.60,0.02}{#1}}
\newcommand{\SpecialCharTok}[1]{\textcolor[rgb]{0.00,0.00,0.00}{#1}}
\newcommand{\StringTok}[1]{\textcolor[rgb]{0.31,0.60,0.02}{#1}}
\newcommand{\VerbatimStringTok}[1]{\textcolor[rgb]{0.31,0.60,0.02}{#1}}
\newcommand{\SpecialStringTok}[1]{\textcolor[rgb]{0.31,0.60,0.02}{#1}}
\newcommand{\ImportTok}[1]{#1}
\newcommand{\CommentTok}[1]{\textcolor[rgb]{0.56,0.35,0.01}{\textit{#1}}}
\newcommand{\DocumentationTok}[1]{\textcolor[rgb]{0.56,0.35,0.01}{\textbf{\textit{#1}}}}
\newcommand{\AnnotationTok}[1]{\textcolor[rgb]{0.56,0.35,0.01}{\textbf{\textit{#1}}}}
\newcommand{\CommentVarTok}[1]{\textcolor[rgb]{0.56,0.35,0.01}{\textbf{\textit{#1}}}}
\newcommand{\OtherTok}[1]{\textcolor[rgb]{0.56,0.35,0.01}{#1}}
\newcommand{\FunctionTok}[1]{\textcolor[rgb]{0.00,0.00,0.00}{#1}}
\newcommand{\VariableTok}[1]{\textcolor[rgb]{0.00,0.00,0.00}{#1}}
\newcommand{\ControlFlowTok}[1]{\textcolor[rgb]{0.13,0.29,0.53}{\textbf{#1}}}
\newcommand{\OperatorTok}[1]{\textcolor[rgb]{0.81,0.36,0.00}{\textbf{#1}}}
\newcommand{\BuiltInTok}[1]{#1}
\newcommand{\ExtensionTok}[1]{#1}
\newcommand{\PreprocessorTok}[1]{\textcolor[rgb]{0.56,0.35,0.01}{\textit{#1}}}
\newcommand{\AttributeTok}[1]{\textcolor[rgb]{0.77,0.63,0.00}{#1}}
\newcommand{\RegionMarkerTok}[1]{#1}
\newcommand{\InformationTok}[1]{\textcolor[rgb]{0.56,0.35,0.01}{\textbf{\textit{#1}}}}
\newcommand{\WarningTok}[1]{\textcolor[rgb]{0.56,0.35,0.01}{\textbf{\textit{#1}}}}
\newcommand{\AlertTok}[1]{\textcolor[rgb]{0.94,0.16,0.16}{#1}}
\newcommand{\ErrorTok}[1]{\textcolor[rgb]{0.64,0.00,0.00}{\textbf{#1}}}
\newcommand{\NormalTok}[1]{#1}
\usepackage{graphicx,grffile}
\makeatletter
\def\maxwidth{\ifdim\Gin@nat@width>\linewidth\linewidth\else\Gin@nat@width\fi}
\def\maxheight{\ifdim\Gin@nat@height>\textheight\textheight\else\Gin@nat@height\fi}
\makeatother
% Scale images if necessary, so that they will not overflow the page
% margins by default, and it is still possible to overwrite the defaults
% using explicit options in \includegraphics[width, height, ...]{}
\setkeys{Gin}{width=\maxwidth,height=\maxheight,keepaspectratio}
\IfFileExists{parskip.sty}{%
\usepackage{parskip}
}{% else
\setlength{\parindent}{0pt}
\setlength{\parskip}{6pt plus 2pt minus 1pt}
}
\setlength{\emergencystretch}{3em}  % prevent overfull lines
\providecommand{\tightlist}{%
  \setlength{\itemsep}{0pt}\setlength{\parskip}{0pt}}
\setcounter{secnumdepth}{0}
% Redefines (sub)paragraphs to behave more like sections
\ifx\paragraph\undefined\else
\let\oldparagraph\paragraph
\renewcommand{\paragraph}[1]{\oldparagraph{#1}\mbox{}}
\fi
\ifx\subparagraph\undefined\else
\let\oldsubparagraph\subparagraph
\renewcommand{\subparagraph}[1]{\oldsubparagraph{#1}\mbox{}}
\fi

%%% Use protect on footnotes to avoid problems with footnotes in titles
\let\rmarkdownfootnote\footnote%
\def\footnote{\protect\rmarkdownfootnote}

%%% Change title format to be more compact
\usepackage{titling}

% Create subtitle command for use in maketitle
\newcommand{\subtitle}[1]{
  \posttitle{
    \begin{center}\large#1\end{center}
    }
}

\setlength{\droptitle}{-2em}

  \title{EEB313 Final Project - swimmyboiz}
    \pretitle{\vspace{\droptitle}\centering\huge}
  \posttitle{\par}
    \author{Dianya Luo}
    \preauthor{\centering\large\emph}
  \postauthor{\par}
      \predate{\centering\large\emph}
  \postdate{\par}
    \date{December 5, 2018}


\begin{document}
\maketitle

\subsection{R Markdown}\label{r-markdown}

This is an R Markdown document. Markdown is a simple formatting syntax
for authoring HTML, PDF, and MS Word documents. For more details on
using R Markdown see \url{http://rmarkdown.rstudio.com}.

When you click the \textbf{Knit} button a document will be generated
that includes both content as well as the output of any embedded R code
chunks within the document. You can embed an R code chunk like this:

\begin{Shaded}
\begin{Highlighting}[]
\KeywordTok{summary}\NormalTok{(cars)}
\end{Highlighting}
\end{Shaded}

\begin{verbatim}
##      speed           dist       
##  Min.   : 4.0   Min.   :  2.00  
##  1st Qu.:12.0   1st Qu.: 26.00  
##  Median :15.0   Median : 36.00  
##  Mean   :15.4   Mean   : 42.98  
##  3rd Qu.:19.0   3rd Qu.: 56.00  
##  Max.   :25.0   Max.   :120.00
\end{verbatim}

\subsection{Including Plots}\label{including-plots}

You can also embed plots, for example:

\includegraphics{EEB313-Final-project-swimmyboiz_files/figure-latex/pressure-1.pdf}

Note that the \texttt{echo\ =\ FALSE} parameter was added to the code
chunk to prevent printing of the R code that generated the plot.

\section{swimmyboiz}\label{swimmyboiz}

\subsection{Abstract}\label{abstract}

This study aimed to examine possible physical and chemical effects on
lake health using the common biological indicators: Ephemeroptera,
Plecoptera and Trichoptera (EPT). As many studies look at anthropogenic
sources and pollutants affecting water health, we aimed to determine how
percent EPT differs based on lake physicochemical variables. The data
used in this study derives from the US Environmental Protection Agency's
2012 National Lakes Assessment. Physical and chemical variables used
include temperature, pH, conductivity, and dissolved oxygen content and
benthic macroinvertebrate data used included percent EPT. Linear scatter
plots with regression lines were created to investigate correlations
between the variables at interest. After no significant correlations,
model selection was performed using a saturated model including the
variables as fixed effects and interactions with temperature as well.
Next, models with delta less than or equal to two were selected as the
top models and model averaging was performed to limit uncertainties
within the top models. Using AIC values and model averaging, the results
indicated that no interactions between temperature and the other
variables had any significant effect on EPT percentage. Only depth and
temperature reported significant effects on percent EPT. This study
suggests there are differences in percent EPT based on lake
physicochemical variables, indicating the physical parameters of the
lake may affect percent EPT. Lake physicochemical must be taken into
consideration, along with current chemical state for water quality
assessments and conservation. The findings of this study and future
studies can be implemented into conservation efforts to ensure the
maintenance and longevity of aquatic ecosystems.

\subsection{Introduction}\label{introduction}

A strong indicator of water quality is the diversity of benthic
macroinvertebrate. Benthic macroinvertebrates are invertebrate taxa
living on the ground of aquatic ecosystems, such as insect larvae,
crustaceans, molluscs, and aquatic worms. The use of these taxa as
indicators to assess water quality of lakes, streams and rivers is
becoming a common method of biological monitoring (Freshwater
Biomonitoring and Benthic Macroinvertebrates). The macroinvertebrate
species with the most restricted tolerances towards water quality
degradation are Ephemeroptera, Plecoptera, and Trichoptera, broadly
termed EPT (Lenat, 1993).

Past research found that physical and chemical properties of aquatic
ecosystems can indicate various EPT biodiversity metrics independently
(Cairns \& Pratt, 1993; Schartau, Moe, Sandin, McFarland, \& Raddum,
2008; Slooff, 1983). The sensitivity of EPT to both physical as well as
chemical parameters result in its robust characteristic as a biological
indicator. The difficulty here lies in dissecting the differences
between physicochemical variables and pollutants on EPT biodiversity.
That is, the EPT biodiversity metric cannot be used alone as an
indicator of anthropogenic impact, as it constantly reflects its
intolerance to physical and chemical variables. This study hypothesizes
physicochemical variables will significant affect percent EPT.

Our study was motivated by the National Lakes Assessment (NLA), an
assessment conducted in 2012 to evaluate the conditions of lakes in the
USA (U.S. Environmental Protection Agency, 2016). The NLA used physical,
chemical and biological indicators to assess the condition of lakes but
did not fully examine the effect of physical and chemical water
qualities on the biological indicators.

Our aims were twofold: first, we examined the effects of physical and
chemical parameters on EPT in lakes; then, we examined the effects of
temperature on the relationship between EPT and abiotic factors. Using
this dataset, we first investigated the effect of five abiotic factors
on EPT: depth, pH, conductivity, temperature and dissolved oxygen. We
determined that it was most important to study the interactive effects
of temperature on other parameters, as benthic macroinvertebrates have
been found to be sensitive to thermal regimes, and small changes will
likely affect the EPT diversity0020(Durance \& Ormerod, 2007; Gore,
1977; Munn \& Brusven, 1991). From this study, we can gain a greater
understanding of how different water qualities in lakes affect benthic
macroinvertebrates.

\subsection{Methods}\label{methods}

\subsubsection{Data description}\label{data-description}

The data for this comes from the U.S Environmental Protection Agency
2012 National Lakes Assessment (EPA, 2016). The EPA collected data from
1,038 lakes in the United States in the year 2012. Field crews collected
water quality and benthic invertebrate data from lakes. For detailed
field data collection techniques, refer to the technical report (US
Environmental Protection Agency (U.S. EPA), 2017). The data from the
2012 National Lakes Assessment report (U.S. Environmental Protection
Agency, 2016) includes measures of algal toxins, atrazine, benthic
macroinvertebrates, chlorophyll-a, physical habitat, landscape data,
phytoplankton, water chemistry, Secchi depth, sediment, site
information, water isotope variables and zooplankton. For this study, we
selected the `Benthic Macroinvertebrate Metrics' and `Water Chemistry'
data sets. The `Benthic Macroinvertebrate Metrics'€™ dataset includes
data on the number and percentage of various taxa found within the
samples. The `€˜Water Chemistry'€™ data set includes data on various
site specific physical and chemical variables. After collecting the
data, we performed all analysis using R 3.5.1 and R Studio 1.1.383.

\paragraph{Data analysis}\label{data-analysis}

First, we examined both data sets and filtered out the variables to be
used in the analysis. From the `€˜Benthic Macroinvertebrate Metrics'€™
data set, we selected columns using the select function in R package
tidyverse to select SITE\_ID (a unique site identifier) and EPT\_PTAX
(the percent of the taxa in the sample which were EPT). Next, we
filtered out all NA'€™s out of the EPT\_PTAX column in the benthic
macroinvertebrate dataset. If no SITE\_ID was recorded, we replaced the
empty cells with NA'€™s using the mutate function and then filtered
NA'€™s from the site column as well.

Within the `€˜Water Chemistry'€™ data set, we selected SITE\_ID, State,
CONDUCTIVITY, DEPTH, OXYGEN, PH, and TEMPERATURE using the select
function in the package tidyverse. There were several water chemistry
samples for each site, so we averaged the water chemistry samples by
site using the mutate and summarize function in the package tidyverse.
Several SITE\_ID'€™s were missing, so we filled the empty cells with
NA'€™s using the mutate function in tidyverse, then filtered all NA'€™s
out of the SITE\_ID column. We also filtered out NA'€™s from the
CONDUCTIVITY, DEPTH, OXYGEN, PH, and TEMPERATURE columns. Next, we
merged the selected and filtered water chemistry and benthic
macroinvertebrate data sets using the right\_join function and grouped
by Site\_ID. Following this, we explored the data and noticed outliers
within the temperature data; therefore, we replaced all temperatures
greater than 50 degrees Celsius with NA and filtered NA'€™s out of the
temperature column.

After the data sets were organized and cleaned, we began to analyze the
data using the package ggplot. EPT\_PTAX was plotted as the response
variable against each water chemistry variable. A linear model
regression line was plotted over the resulting plots to investigate
correlations in the data. As no clear correlation emerged from any of
the plots independently, we performed model selection using the package
MuMIN. We created a saturated model with the variables as fixed effects
and interactions of temperature with depth, oxygen and conductivity as
depth, oxygen and conductivity showed significant differences based on
EPT\_PTAX. This was done to investigate if temperature interactions
change relationships between EPT and the other variables. Next, we
dredged the model using AIC as the rank through the function dredge.
Then, we subset the models produced from model dredging to models with a
delta less than or equal to two. After subsetting, only four models
remained, and these models were averaged to arrive at the final model.
This is done as any model with a delta less than or equal to 2 is not
significantly different, thus averaging contributes to avoiding
uncertainty in the models. All R code used for this project is attached
at the end of this paper.

\subsubsection{Results}\label{results}

Figure 1. Scatter plot depicts the averaged temperature (°šC) at each
site against the \% of EPT taxa found at each site. The blue line
represents the regression line, which exhibits weak to no correlation
between the two variables.

Figure 2. Scatter plot depicts the dissolved oxygen at each site against
the \% of EPT taxa found at each site. The blue line represents the
regression line, which exhibits weak positive correlation between the
two variables.

Figure 3. Scatter plot depicts the pH at each site against the \% of EPT
taxa found at each site. The blue line represents the regression line,
which exhibits weak to no correlation between the two variables.

Figure 4. Scatter plot depicts the depth at each site against the \% of
EPT taxa found at each site. The blue line represents the regression
line, which exhibits a weak positive correlation between the two
variables.

Figure 5. Scatter plot depicts the conductivity at each site against the
\% of EPT taxa found at each site. The blue line represents the
regression line, which exhibits a weak negative correlation between the
two variables.

Variable Estimate Std. Error t value
Pr(\textgreater{}\textbar{}t\textbar{}) Adjusted R-squared Depth
5.29E-01 6.54E-02 8.078 1.832-15 0.04627 Conductivity -1.52E-04 5.30E-05
-2.87 0.004183 0.009016 Temperature 1.60E-01 4.43E-02 3.605 0.000328
4.06E-05 Oxygen 2.69E-01 8.44E-02 3.184 0.001495 0.00495 pH -1.64E-01
2.18E-01 -0.75 0.453202 0.2616

Table 1: Table depicting results from first general linear model,
including all chemical and physical variables of interest. Bolded were
the variables that were found to be most significant. Adjusted R-squared
for each variable was calculated separately in R in a general linear
model with EPT\_PTAX as the dependent variable. Statistical significance
at P \textless{} 0.05 is indicated by bold text.

Model df logLik AIC delta weight Adjusted R-squared Model with No
Interactions 6 -3351.59 6715.19 0 0.39 0.07298 Model with Conductivity
and Temperature Interactions 7 3351.02 6716.03 0.85 0.25 0.07313 Model
with Oxygen and Temperature Interactions 7 -3351.35 6716.7 1.51 0.18
7.24E-02 Model with Depth and Temperature Interactions 7 -3351.37
6716.73 1.55 0.18 0.07257

Table 2: Four top models extracted from the dredged saturated model
created with interactions between temperature and the three other
significant variables. In bold is the most significant model with the
lowest AIC value.

Estimate Std. Error Adjusted SE z value
Pr(\textgreater{}\textbar{}z\textbar{})

(Intercept) 4.384069e+00 1.623845e+00 1.625513e+00 2.6970371 0.0069959
Conductivity -9.703185e-05 1.623372e-04 1.624494e-04 0.5973052 0.5503036
Depth 5.135467e-01 1.042203e-01 1.043229e-01 4.9226652 0.0000009 Oxygen
2.885777e-01 1.741035e-01 1.742630e-01 1.6559901 0.0977238 Temperature
1.705990e-01 6.867057e-02 6.873744e-02 2.4818939 0.0130686
Conductivity:Temperature -3.163551e-06 7.996480e-06 8.001552e-06
0.3953671 0.6925720 Oxygen:Temperature -1.832567e-03 7.299212e-03
7.305366e-03 0.2508522 0.8019284 Depth:Temperature 1.300781e-03
5.361594e-03 5.366199e-03 0.2424026 0.8084682 Table 3: Model averaged
coefficients, the full averaged table extracted. The terms bolded were
the significant variables indicated in R. Statistical significance at P
\textless{} 0.05 is indicated by bold text.

As the first step in our study, Figures 1-5 depict our exploratory data
analysis, where we attempted to identify any correlation between EPT
percentage with the chemical and physical variables of interest; oxygen,
pH, temperature, conductivity and depth. There was weak correlation in
each of the graphs as depicted by the linear regression line, but slight
trends were found between depth, dissolved oxygen and conductivity.

Further analysis was conducted by creating a saturated model including
these terms together, which provided the variables that had significant
effect on EPT percentage (Table 1). Depth had the most significant
effect with a p-value of 1.832e-15, and exhibited unexpected results, as
depth and EPT percent had a positive correlation. The other parameters
that were deemed significant in the basic linear model were conductivity
and oxygen (Table 1). It was found that as conductivity increased EPT
percentage decreased, and when oxygen increased EPT percentage also
increased. There was no significant effect seen for pH (p = 0.453), as
shown in Table 1.

Next, we further investigated temperature'€™s effect on the other
abiotic, where interactions between temperature and the other
significant variables were added to the saturated model and dredged to
obtain the top models. As seen in Table 2, the model with no
interactions produced the lowest AIC value, indicating it was the most
significant. Using averaging, depicted in Table 3, we found no
significant interaction between temperature and the other variables.
Additionally, model averaging narrowed the significant predictors down
to two: only depth and temperature had a significant effect on EPT
percentage.

\subsubsection{Discussion}\label{discussion}

In this study, our aim was to examine the effects of the physical and
chemical parameters on EPT in lakes, and the interaction of temperature
on the significant abiotic factors on EPT taxa percentage. Within the
main effects on EPT percent taxa, we found that depth had the most
significant effect on EPT percent taxa, followed by temperature, oxygen
and conductivity respectively.

Interestingly, pH had no significant effect on EPT, as the graph in
Figure 3 shows no correlation. According to Schartau et al. (2008), the
optimum pH levels for EPT are between 5.8 and 6.5; most
macroinvertebrates can sustain in water closer to a neutral pH but
rarely when acidic (\textless{}5.8). According to Figure 3, it appears
that the frequency of data points depicting EPT fall in environments
between a pH level of 6.0 to 9.0. Beyond these ranges, EPT percent
diversity should decrease and be less common among fresh bodies of water
at those levels. The larger variance at pH levels less than 6.0 and
greater than 9.0 indicate that either there are less lakes with these pH
levels, or EPT diversity is less likely to be observed at these poorer
conditions. However, it must be noted that pH was a nonsignificant
indicator of EPT diversity, and so these results are not considered
crucial in our results.

Depth had the largest effect on EPT percent taxa, but not in the pattern
that we expected. We predicted that depth would have a large effect as
these three biological indicator species are nymphs in the aquatic
system, and therefore need close access to the terrestrial environment.
However, the positive correlation between depth and EPT diversity was
not expected. EPT taxa typically prefer shallow lakes to breed,
according to previous literature (Jerves-Cobo et al., 2017). This could
be due to the limitations of our methods: we averaged the depth of lakes
and recorded this average depth, instead of using the depth at locations
where the EPT taxa were observed and reported. Additionally, as depth is
an inherent quality of lakes, depth could be a confounding variable. EPT
may prefer lakes of specific depths that most likely does not originate
from an anthropogenic source or pollutant.

EPT percent taxa decreased as conductivity increased but this
interaction was not significant. According to the EPA, high conductivity
is indicative of high salinity, resulting in poor water quality that is
not tolerated by many aquatic organisms. High conductivity is often the
result of human disturbances through chemical introduction such as
agricultural run-off. As seen in Figure 5, most of the data points are
concentrated at the 0 Siemens per meter, which implies that there was
little influx of inorganic nutrients within most of the sites tested,
potentially resulting in the non-significance of conductivity.

Increasing dissolved oxygen (DO) levels were shown to increase EPT
diversity but it was not significant either. The EPA states that levels
of DO below 3 mg/L are of concern, and those below 1 mg/L are considered
hypoxic and are usually devoid of life (. As shown in Figure 2, most
data points are concentrated between 2.5 mg/L and 10 mg/L indicating
moderately healthy lakes. However, EPT appears to have a broad
sensitivity to this chemical property which can be deduced from the
nonsignificant result, and thus we question its credibility as a
predictor for EPT diversity in an aquatic ecosystem.

We also observed a broad spectrum of viability in terms of EPT percent
taxa in response to temperature. The lakes with a temperature greater
than 25°šC appear to be unlikely to have abundant EPT, although they are
presented to be abundant. The methods used to collect temperature leads
us to be skeptical of the results. The overall trend presented shows
that as temperature increases, EPT \% taxa increases, which goes against
our predictions and previous literature. It was believed that EPT would
have a narrow sensitive range to temperature changes, as warmer
temperatures usually lead to lower DO levels.

Finally, the interactions between temperature and the other abiotic
factors of conductivity, oxygen, and depth showed no significance. Our
hypothesis that temperature would have a significant impact on the other
abiotic factors by negatively impacting the EPT taxa percentage was
rejected by this result. That is, temperature does not change the
abiotic variables in a way that significantly predicts EPT diversity.

However, notably, dissolved oxygen content behaves differently when
temperature was added to the model; it resulted in an interaction which
causes EPT percent taxa to decrease, as opposed to increase when
temperature was not in the model. This could be due to pressures that
temperatures exert on biological processes of aquatic species, such as
predation (Moore \& Townsend, 2017) and growth (Cuenco, Stickney, \&
Grant, 1985). Temperature was a significant limiting variable for EPT
diversity; coupled with the evidence that greater aquatic activity
requires greater rates of oxygen intake, this interaction could generate
a narrower optimal range for EPT, indicating poorer water conditions.
Although this interaction was not significant, the dramatic reversal of
the effect of DO after adding temperature as an interacting variable is
an area of further research.

As there were no significant effects found in using interacting abiotic
variables, this brought the limitations of this study to light. There
are many different metrics used in this study and other studies as
indices of water quality and lake ecosystem health, and no method of
selecting the most optimal one. We did not consider location as an
element that could affect EPT diversity. The proximity to urban,
agricultural or terrestrial protected areas could influence EPT
diversity, as well as other physical and chemical parameters.
Furthermore, the difficulty in using benthic macroinvertebrates as
biological indicators revolves around subjective scoring methods of
their tolerance, and the many criteria for measuring their diversity
(Cairns \& Pratt, 1993). Different taxa have different responses to
changes in the aquatic ecosystem, and no single index of diversity has
been shown to be the most statistically powerful measure (Slooff, 1983).
Until standardized metrics of macroinvertebrate tolerance and diversity
are used, classifications of water quality will be unreliable, and the
literature will be inconsistent.

\subsection{Conclusions}\label{conclusions}

This study found that benthic macroinvertebrate biodiversity was
significantly affected by depth, conductivity, temperature and dissolved
oxygen content, but not pH in lakes in the USA. When graphed at first,
it appeared that only weak correlations existed between EPT diversity
and the physical and chemical variables. Using a linear regression
model, we found that depth, conductivity, temperature and dissolved
oxygen significantly affected the diversity of EPT in lakes.

Furthermore, we found that there was no significant interaction between
the variables. Independently, a greater depth, lower conductivity,
greater temperature, and greater dissolved oxygen content appear to be
the best conditions for greater EPT percent diversity, which in turn may
indicate water quality. However, interacting variable terms did not
significantly improve the model.

Based on our study, we conclude that physical parameters, not just
pollution, need to be examined for conservation efforts. There are other
characteristics of a lake that are not a result of pollution, such as
depth. Researchers need to be careful when using different sites for
comparison and concluding that different sites have different water
qualities without dissecting all variables involved, such as location,
presence of urban areas, and proximity to protected areas.

Future studies that examine water quality need to ensure that more than
just broad-scale patterns in biological indicator richness are used to
assess water quality. Water quality research should explore known
physical and chemical pollutant and the pollutants'€™ effect on EPT
diversity. Here, we only had the resources to examine five of the most
affecting variables, and they were not all pollutants. The effect of
pollutants and their interactions may result in larger changes in EPT
diversity, resulting in more severe consequences. The findings of this
study and future studies can be implemented into conservation efforts to
ensure the maintenance and longevity of aquatic ecosystems.

\subsection{References}\label{references}

Cairns, J., \& Pratt, J. R. (1993). A history of biological monitoring
using benthic macroinvertebrates. Freshwater Biomonitoring and Benthic
Macroinvertebrates, 10, 27. Cuenco, M. L., Stickney, R. R., \& Grant, W.
E. (1985). Fish bioenergetics and growth in aquaculture ponds: II.
Effects of interactions among, size, temperature, dissolved oxygen,
unionized ammonia and food on growth of individual fish. Ecological
Modelling, 27(3'€``4), 191'€``206.
\url{https://doi.org/10.1016/0304-3800(85)90002-X} Durance, I., \&
Ormerod, S. J. (2007). Climate change effects on upland stream
macroinvertebrates over a 25-year period. Global Change Biology, 13(5),
942'€``957. \url{https://doi.org/10.1111/j.1365-2486.2007.01340.x} Gore,
J. A. (1977). Reservoir manipulations and benthic macroinvertebrates in
a Prairie River. Hydrobiologia, 55(2), 113'€``123.
\url{https://doi.org/10.1007/BF00021052} Jerves-Cobo, R., Everaert, G.,
Iñiguez-Vela, X., Córdova-Vela, G., D�az-Granda, C., Cisneros, F.,
`€¦ Goethals, P. L. M. (2017). A methodology to model environmental
preferences of EPT taxa in the Machangara River Basin (Ecuador). Water
(Switzerland) (Vol. 9). \url{https://doi.org/10.3390/w9030195} Lenat, D.
R. (1993). A Biotic Index for the Southeastern United States: Derivation
and List of Tolerance Values, with Criteria for Assigning Water-Quality
Ratings. Journal of the North American Benthological Society, 12(3),
279'€``290. \url{https://doi.org/10.2307/1467463} Moore, M. K., \&
Townsend, V. R. (2017). The Interaction of Temperature, Dissolved Oxygen
and Predation Pressure in an Aquatic Predator-Prey System. Oikos, 81(2),
329'€``336. Munn, M. D., \& Brusven, M. A. (1991). Benthic
macroinvertebrate communities in nonregulated and regulated waters of
the clearwater river, Idaho, U.S.A. Regulated Rivers: Research \&
Management, 6(1), 1'€``11. \url{https://doi.org/10.1002/rrr.3450060102}
Schartau, A. K., Moe, S. J., Sandin, L., McFarland, B., \& Raddum, G. G.
(2008). Macroinvertebrate indicators of lake acidification: Analysis of
monitoring data from UK, Norway and Sweden. Aquatic Ecology, 42(2),
293'€``305. \url{https://doi.org/10.1007/s10452-008-9186-7} Slooff, W.
(1983). Benthic macroinvertebrates and water quality assessment: Some
toxicological considerations. Aquatic Toxicology, 4(1), 73'€``82.
\url{https://doi.org/10.1016/0166-445X(83)90062-0} U.S. Environmental
Protection Agency. (2016). National Lakes Assessment 2012: A
Collaborative Survey of Lakes in the United States, (December), 40.
Retrieved from \url{https://nationallakesassessment.epa.gov/} US
Environmental Protection Agency (U.S. EPA). (2017). National Lakes
Assessment 2012: Technical Report, (April), 168. US Environmental
Protection Agency. `€œIndicators: Conductivity.'€ United States
Environmental Protection Agency. National Aquatic Resource Surveys, 16
August 2016,
\url{https://www.epa.gov/national-aquatic-resource-surveys/indicators-conductivity}.
US Environmental Protection Agency. `€œIndicators: Dissolved Oxygen.'€
United States Environmental Protection Agency. National Aquatic Resource
Surveys, 16 August 2016,
\url{https://www.epa.gov/national-aquatic-resource-surveys/indicators-dissolved-oxygen}.


\end{document}
